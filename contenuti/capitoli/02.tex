\chapter{Ricerca}
L'ottanta percento delle madri tiene il figlio dalla parte sinistra. Questo è istintivamente dovuto al fatto che il bambino si sente maggiormente rassicurato se sente il battito cardiaco della madre, visto che per mlti mesi ha vissuto sentendolo continuamente. Come dato, anche l'ottanta percento dei pittori ritrattisti ritrae il soggetto con la tre quarti sinistra del volto verso l'osservatore. Questo è dovuto al fatto che, come nella manipolazione, anche nella mimica facciale c'è una lateralizzazione, quindi la parte sinistra del volto tende ad essere più espressiva. I pittori, per istinto scelgono la parte migliore e la espongono all'osservatore.

Nelle paralisi del faciale, quando c'è una paralisi della parte più espressiva, la parte meno espressiva, che deve supplire alla mimica maggiore persa, tende a tirare tutto il volto dalla sua parte. Questo ci obbliga a chiederci, inizialmente, quale sia la parte colpita.

Sperimentazione e ricerca sono comunque due cose diverse: la ricerca si avvale di tanti metodi, di tanti processi logici. All'interno di questi, si distingue la ricerca osservazionale dalla ricerca sperimentale, dalla ricerca qualitativa.

Nella ricerca osservazionale, lo sperimentatore non interviene a manipolare il fenomeno, lo lascia com'è e lo osserva, mentre nella ricerca sperimentale si introduce una variabile, si modifica il fenomeno che vado a misurare.

Classicamente, gli studi osservazionali sono quelli
\begin{itemize}
\item trasversali
\item caso-controllo
\item di coorte
\end{itemize}
Negli studi trasversali si verifica la prevalenza puntuale di un dato fenomeno: è una fotografia di una popolazione più o meno ampia, che dovrebbe idealmente tendere all'intera popolazione.

Negli studi caso-controllo si selezionano le persone che presentano una certa malattia (i casi) e quelli che non la presentano (i controlli), per poi verificare quanti casi e quanti controlli presentino un certo fattore. \'e uno studio puramente osservazionale, di cui siamo certi, perché questo tipo di studio è retrospettivo. In questo tipo di studi si ricava, alla fine, l'\textit{odds ratio}, che oscilla attorno a 1. Se questo valore è esattamente 1, siamo sicuri che non ci sia correlazione tra il fattore di rischio e la malattia, se è maggiore di 1 abbiamo una correlazione, più o meno forte, tra la malattia e il fattore di rischio, mentre se è minore di uno, il fattore viene detto \textit{protettivo}.
Questo tipo di studi può essere affrontato tramite interviste o tramite documentazione clinica.

Gli studi di coorte sono studi prospettici: si sceglie di iniziare a studiare una certa malattia oggi, e la si segue per anni, o si introduce un nuovo farmaco, e si segue la sua evoluzione per alcuni anni. Gli studi di coorte sono molto più controllabili di quelli caso-controllo, perché il clinico ha proprio il controllo, momento per momento, dei pazienti.

Questi studi possono avere degli aspetti positivi o negativi, detti BIAS. I BIAS possono essere di selezione (come viene reclutato il campione), di misura (quelli legati alla misurazione della variabile indipendente), o di confondimento (essenzialmente sono le variabili intervenienti, che confondono la ricerca perchè non vengono considerati in anticipo). Questi vanno ad analizzare aspetti temporali, economici, e di attendibilità.

Gli studi trasversali sono brevi, semplici, poco costosi, ma non permettono di stimare correttamente fenomeni poco frequenti, e non sono accurati per il fattore di rischio.

Gli studi caso-controllo sono anche questi brevi, sono adatti per le malattie rare, e %%%

Gli studi di coorte sono attendibili per il rischio relativo, ma sono lunghi, costosi, hanno difficoltà tecniche (perdita dei casi, variazione dell'esposizione, cambiamenti delle tecniche diagnostiche, \dots), devono avere un numero elevato di soggetti, che spesso non è facile da ottenere, e non sono adatti, ovviamente, per le malattie rare. Le differenze diagnostiche, in questi studi, sono molto importanti, perché la maggiore diagnosticabilità di una certa malattia rischia di rendere irrilevante una ricerca fatta su tecniche vecchie.

\section{Disegno sperimentale}
\begin{itemize}
\item Definizione del problema: Perché i fisioterapisti manifestano determinate patologie, ad esempio il dito a scatto?
\item Formulazione dell'ipotesi: Il mancato utilizzo di ausili in ambito ortopedico può portare ad una maggiore comparsa del dito a scatto?
\item Scelta delle variabili:
\begin{itemize}
\item Variabile indipendente: Utilizzo di ausili/utensili appropriati.
\item Variabile dipendente: Variazione dell'incidenza del dito a scatto.
\end{itemize}
\item Controllo variabili intervenienti:
\begin{itemize}
\item Scelta del campione: Scegliamo gli studenti del terzo anno, maschi, che scelgono di dedicarsi alla terapia ortopedica.
\item Gruppo di controllo: La parte di fisioterapisti a cui non assegniamo un ausilio.
\item Cieco: Non informiamo i terapisti sul vero scopo della ricerca, facendogli credere che stiamo eseguendo una ricerca sull'efficacia dell'ausilio nel trattamento del paziente, non nella prevenzione della patologia professionale.
\item Doppio cieco: Lo statistico a cui affideremo i dati da elaborare non sarà informato sul tipo di ricerca.
\end{itemize}
\item Fase sperimentale: 
\item Raccolta dati: Eseguita con un questionario con item multipli, in modo da non far capire che la ricerca verte sull'efficacia nella prevenzione.
\item Elaborazione dati: Eseguita da uno statistico, appositamente accecato.
\item Valutazione
\end{itemize}