\mainmatter
\chapter{Discorso sul metodo}
Il metodo è un processo logico, ciascuno ha il suo personale, che deve essere costruito. Chi non ha un metodo, va a caso, per tentativi
ed errori, il che non è professionale. L'esempio può essere la costruzione di un tavolo: si può andare alla cieca al negozio, o pensare
prima a cosa serve, in modo da fare un solo viaggio, o la preparazione della pasta.

Il metodo formale della scienza è ispirato da Galileo Galilei, ma un tale Popper diceva che la scienza nasce da un problema. Nel nostro
schema manca il problema. 
\begin{large}
Il metodo sperimentale
\end{large}
\begin{itemize}
\item Individuazione del problema
\item Formulare un'ipotesi/tesi
\item Scelta delle variabili
\begin{itemize}
\item Variabile indipendente
\item Variabile dipendente
\end{itemize}
\item Controllo delle variabili intervenenti
\begin{itemize}
\item Scelta del campione
\item Gruppo di controllo
\item Cieco
\item Doppio cieco
\end{itemize}
\item Fase sperimentale
\item Raccolta dati, con misurazione della variabile dipendente
\item Elaborazione dei dati
\item Valutazione
\end{itemize}
Nell'articolo dei topi, il problema è che l'apprendimento del topo si produce anche in età adulta: secondo la teoria classica, il
cervello diventa immutabile dopo l'infanzia, ma questo dovrebbe rendere impossibile l'apprendimento. La formulazione del problema,
quindi, permette di porsi la domanda giusta, su cui impiantare il proprio lavoro.

\section{Ricerche sulle paralisi del faciale}
Nelle ricerche del prof. \myProf, il problema è nato dal fatto che, in seguito alle paralisi del faciale, \textit{che tradizionalmente
è un nervo solo motore}, nascevano delle sincinesie. Questo lo ha portato a chiedersi da cosa derivassero, e perché fossero più presenti
dopo alcuni mesi piuttosto che in fase acuta. Le lesione del faciale derivano spesso e volentieri da una irritazione iniziale,
che tramite l'edema schiaccia e provoca danni secondari al nervo, che va in sofferenza. In seguito alla sofferenza assonale, poi, può
andare in sofferenza anche la parte centrale della cellula, che muore. Recentemente è stata scoperta anche la degenerazione
\textit{trans-neuronale} o \textit{trans-sinaptica}. Nei nervi cranici, che vengono studiati come dodici paia, ci sono, anche dal punto
di vista anatomico, numerose anastomosi, con cui i nervi si scamiano fibre. Questo porta, con la degenerazione del settimo nervo cranico,
ad una degenerazione degli altri, che sono collegati. Dopo una paralisi, e dopo 15 giorni di processi degenerativi, la struttura si
trova ad essere, anche fisicamente, molto ridotta. Per questo, in seguito, ci devono essere dei processi rigenerativi, che devono
reinnervare le zone denervate. Una reinnervazione non precisa, purtroppo, può portare ad un instradamento sbagliato delle fibre, che
porta alla formazione di sincinesie patologiche.

Gli esercizi motori, in fase acuta, possono essere addirittura interferenti in senso negativo con la reinnervazione. Si eseguono quindi
degli esercizi per la sensibilità, estero e propriocettiva, per evitare le sincinesie, ma cercando di non interferire con la
reinnervazione.

Nelle ricerche della Montalcini si sono individuati dei fattori trofici, che orientano la crescita del neurone. Questi fattori sono
anche indispensabili per la sopravvivenza del neurone, e vengono secreti dai bersagli del neurone. Il fattore trofico, quindi, permette
di orientare il cono di accrescimento e la crescita dell'assone verso cellule bersaglio specifiche. Il fattore trofico è molto presente
nella fase dello sviluppo, per poi scomparire quando il SNC giunge a maturazione. Ricompare quando ci sono dei danni al SNC,
probabilmente per la sua necessità di riorganizzarsi.

La cosa importante, alla fine, è che ogni processo di apprendimento è dato da un cambiamento della nostra rete neuronale, che è quindi
in continua permutazione. Le nuove modificazioni della rete neuronale possono essere fissate, strutturate, se l'esperienza viene
ripetuta, o perse.

Piaget sosteneva che noi ricordiamo poco delle nostre esperienze da bambini perchè i nostri sistemi cognitivi, nel tempo, si sono
sovrapposti.

Tornando al discorso del movimento da evitare, si è provato, sperimentalmente, a prendere un neurone tagliato e una fibra muscolare. 
Lasciando ferma la cellula muscolare, o addirittura inibirla, provoca un aumento del numero di motoneuroni che sopravvivono. Viceversa, 
stimolare la cellula muscolare porta alla morte di un più alto numero di motoneuroni. Questo è dovuto al fatto che i neuroni dipendono 
dai loro bersagli: se la fibra non si contrae, inizia a secernere fattore neurotrofico, per portare a una riorganizzazione della rete 
neuronale e riottenere una contrazione. In una paralisi del faciale, quindi, la somministrazione precoce di esercizi motori, a maggior 
ragione visto che gli esercizi non stimolano completamente tutte le fibre, provocherebbe una non secrezione di fattore neurotrofico, o 
peggio una secrezione non uniforme. La secrezione non uniforme, maggiore in alcune zone che in altre, porta alla formazione di 
sincinesie, dovute alla reinnervazione aberrante.

\section{Esperimento sui topi}
L'ipotesi è che l'esperienza permetta una modificazione del cervello del topo. Le variabili indipendenti, nell'esperimento, sono date 
dal tipo di gabbia, quindi dal tipo di stimolazione a cui i topi venivano sottoposti: le gabbie potevano essere standard, deprivanti o 
stimolanti.
Le variabili dipendenti che sono state esaminate sono: variazione del peso delle sezioni cerebrali, variazione dei livelli di 
secrezione enzimatica, variazione del rapporto della presenza di DNA e RNA.

Come vairiabili intervenenti si è selezionato il campione, tramite incroci tra fratelli e selezione dei topi partecipanti 
all'esperimento: i ratti discendono da incroci tra fratelli per eliminare il più possibile la variabilità genetica che predisponga alla 
variazione del cervello. I partecipanti all'esperimento erano solo topi maschi, e i criteri di inclusione ed esclusione dovevano essere 
gli stessi per il gruppo sperimentale e quello di controllo, per evitare differenze in partenza. Il gruppo di controllo è particolare: 
un gruppo di topi funge da gruppo di controllo per l'altro gruppo. 
I primi ciechi in questo esperimento sono i topi, che non sanno di prendere parte ad un esperimento. Un altro cieco è dato da chi 
esegue le sezioni nel cervello, che non sa a che gabbia appartenga ogni ratto, ed eventualmente può essere reso cieco anche chi elabora 
i dati, non informandolo sul perchè elabora.

Nella fase sperimentale si descrive lo svolgimento dell'esperimento. Si riportano quindi il tipo di variazioni nella massa del 
cervello, e le variazioni enzimatiche. La variazione del rapporto DNA/RNA è un dato inatteso: va semplicemente indicato, ma non 
discusso nelle conclusioni: essendo i dati emersi durante l'esperimento, si rischia di introdurre dei dati in favore delle ipotesi, 
che vengano manipolati, anche inconsapevolmente, dal ricercatore, per supportare le sue ipotesi. Si rischia anche la creazione di 
ipotesi \textit{ad hoc}.

L'elaborazione dei dati viene affidata, di solito, ad uno statistico, molto più affidabile di noi.

nella valutazione ci si esprime circa la accettazione o il rifiuto dell'ipotesi di partenza, in base ai dati emersi durante la 
trattazione del problema.

\section{Tesi sperimentale}
In una tesi sperimentale, quinid, il primo capitolo funge da introduzione, con la descrizione del problema e la scelta dell'ipotesi. 
La scelta del campione, i materiali utilizzati, vengono esposti in un capitolo successivo, \textit{materiali e metodi}, appositamente 
dedicato. Alla fine si trova la fase dei risultati, che erano stati dichiarati come aspetti da indagare all'inizio, e vengono 
presentati anche i dati inattesi, a cui ci siamo trovati davanti durante l'esecuzione dell'esperimento.
L'ultimo capitolo è quello dedicato alla discussione dei risultati.

La tesi sperimentale, quindi, non è così difficile da fare, il problema più grande è trovare un problema che sia interessante, che 
valga la pena studiare. La maggior parte della pubblicazione di ricerca, dopo pochi anni dalla pubblicazione, viene cestinato, in 
quanto proprio non ha rilevanza o conseguenze a livello operativo. Quello che ci si aspetta dallo studente, quindi, è non tanto che 
produca una tesi che rivoluzioni il mondo scientifico, ma piuttosto che segua in modo rigoroso il metodo: devono essere analizzate la 
variabili che si sono dichiarate, e le conclusioni devono discendere dall'analisi di queste variabili.
