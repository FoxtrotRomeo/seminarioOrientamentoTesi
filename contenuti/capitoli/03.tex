\chapter{Ricerca qualitativa}
\'E una parte della ricerca, e viene utilizzata molto in ambito sociale e psicologico. Nell'800, in periodo positivista, quando l'unico metodo valido era quello scientifico, e ciò che non poteva essere conosciuto con il metodo scientifico non era valido, non veniva considerata. Nel '900, però, ci si rende conto che non si può applicare in ambito sociale quello che è vero per le scienze naturali, e questo ha portato al suo sviluppo.

Per ricerca qualitativa si definisce un processo di indagine che si basa sulla comprensione di un fenomeno sociale, all'interno del suo contesto. Il setting di indagine, quindi, è quello naturale, nell'ambiente di origine del fenomeno. Risponde alle domande "cos'è X e come varia", non a "quanto è X e quanti X ci sono", come la ricerca quantitativa.

Si indaga nell'ambiente naturale del fenomeno, cercando di comprendere cosa il fenomeno rappresenta per le persone. Gli obiettivi sono la comprensione del fenomeno, la responsabilizzazione degli individui, la comprensione empatica del contesto.

Le differenze rispetto alla quantitativa sono già nel disegno della ricerca: nella quantitativa il disegno è ben strutturato, con valutazione di dati numerici e revisione della letteratura, e schema di indagine chiuso e rigido, mentre nella qualitativa il disegno è strutturato, ma molto flessibile, con varie tappe collegate fra loro che ci permettono di tornare indietro, riprogrammare e testare. Lo scopo non è misurare e descrivere, ma spiegare e comprendere.
L'obiettivo della quantitativa è l'oggettività, mentre nella qualitativa viene riconosciuta la soggettività, con una ricerca co-costruita.
Il campione della quantitativa deve essere statisticamente significativo, quindi in numero sufficiente per avere dati significativi. Nella qualitativa non è importante il numero di casi selezionati, ma la probabilità che i casi indagati possano rispondere nel modo migliore alle nostre richieste: ad esempio, in una ricerca sul post-ictus, non ci interessa selezionare 200 soggetti, ma magari ne bastano 5, che rispondano in maniera esaustiva alle nostre richieste, perché esiste il fenomeno della saturazione: nel sentire storie personali si rischia di sentire sempre le stesse cose, in modo che una nuova storia non aggiunga nulla di nuovo a quello che già so.
Il metodo di ricerca è eterogeneo: a seconda del tipo dei partecipanti posso usare un questionario, una intervista, \dots

I dati che si ricavano non sono standardizzati, possono essere brani, sbobine, questionari, pinei di considerazioni personali.
Riguardo alla metodologia, la quantitativa usa un campionamento casuale, per ottenere risultati numerici. Nella qualitativa, il campionamento non è casuale, perché cerchiamo i partecipanti che ci possano dare le risposte migliori.
L'oggetto dell'analisiè il vissuto dell'individuo, i suoi sentimenti, le sue motivazioni.
Nella ricerca quantitativa si vuole spiegare il perché di certi dati, mentre nella qualitativa si vuole comprendere gli individui e le loro percezioni.
La statistica è utilizzata, ma solo come supporto, non come strumento di analisi principale. La presentazione dei dati non viene realizzata con una tabella, ma con delle informazioni descrittive sulla base delle nostre indagini.
Per generalizzare i dati, in quantitativa si esegue una correlazione delle variabili, mentre nella qualitativa si classifica tutto in base alle tipologie dei soggetti che partecipano.

I risultati della quantitativa sono generalizzabili e riproducibili, mentre nella qualitativa è difficile generalizare, anche se si chiede che siano, almeno in parte, riproducibili.
I concetti della ricerca quantitativa sono definitivi, mentre nella ricerca qualitativa sono sensibilizzanti.

La ricerca qualitativa può essere utilizzata prima della quantitativa, per orientare al meglio la successiva quantitativa, con un questionario o un focus group, che ci permettano di ideare una quantitativa con un programma di azioni che più probabilmente ci darà successo.
Può essere utilizzata anche assieme, per analizzare perché un certo fenomeno si verifichi in un certo contesto. Può essere utilizzata anche quando la ricerca quantitativa non riesca ad analizzare il fenomeno: in un ambito nuovo, ad esempio, per cercare indirizzi su come eseguire un certo intervento, si possono valutare tutti i punti di vista in gioco. 

Le fasi della ricerca sono:
\begin{itemize}
\item Scegliere e definire il problema: Mentre per la ricerca quantitativa, il problema di solito lo abbiamo ben chiaro, qui è sfumato. L'argomento e l'idea sono inizialmente vaghi e nebulosi. Innanzitutto si esegue una revisione in letteratura, perché bisonga sempre fare riferimento a quello che è stato scritto. In seguoto si analizzano delle prospettive di ricerca, e formulando delle domande specifiche definiamo il campo di studi.
\item Formulare l'obiettivo cognitivo: Si formula tenendo conto dei fattori che condizioneranno il processo di ricerca. Dobbiamo cercare di restringere il più possibile la nostra indagine. Andando avanti, emergeranno delle domande sempre più specifiche. Per il bene della scienza, una volta definito un obiettivo cognitivo, quello deve restare lo stesso.
\item Definizione delle strategie operative: Gli strumenti di indagine utilizzati: possono essere focus group, questionari, interviste strutturate o non strutturate. Dobbiamo anche scegliere dei soggetti pertinenti.
\item Raccolta dati: Provando gli strumenti sul campo potrebbero sorgere dei problemi: il campione non risponde alle nostre richieste (in questo caso lo cambiamo), gli strumenti non sono idonei (li cambiamo) o le informazioni non sono complete (somministriamo anche un altro strumento). \'E sempre possibile tornare indietro, quindi. \'E importante ricordare che le ipotesi possono sempre essere confutate durante la ricerca.
\item Analisi dei dati: Diventa molto importante il consenso informato, analizzando il vissuto delle persone. Per analizzare, il ricercatore si trova a prendere degli appunti durante la conduzione dell'intervista, che possono essere sia oggettivi, una sorta di riassunto dell'intervista, sia che mettano in evidenza gli atteggiamenti messi in atto durante l'intervista.
\item Presentazione e interpretazione dei risultati: Alla luce del lavoro fatto su ogni intervista, sia dei punti salienti, sia dei nostri commenti, possiamo confrontare tutti questi dai, per chiarire quello che ci eravamo proposti di chiarire, generalizzando o delimitando le conclusioni. Alcuni dati vanno censurati, al contrario di quello che sucede in quantitativa.
\end{itemize}

\section{I criteri}
Nonostante la nebulosità della ricerca qualitativa, esistono dei criteri da rispettare durante l'esecuzione della ricerca. 
Gli obiettivi e le ipotesi di ricerca devono essere ben chiari e definiti.
Il campione deve essere scelto nel modo migliore, cosiccome le tecniche e gli strumenti che andremo a utilizzare.

I dati che otterremo devono essere pertinenti, validi e attendibili, e possono essere raccolti utilizzando tutti gli strumenti a nostra disposizione.
Le informazioni che raccogliamo devono essere trattate, e devono restare disponibili per chi magari volesse esaminare il nostro studio.
Le conclusioni devono concordare con i risultati che otteniamo, e devono restituire una visione d'insieme dell'oggetto che viene indagato.
I risultati devono essere pertinenti ai dati, e la conclusione deve esporre in maniera chiara, comprensibile, e con un linguaggio scientifico.

\section{I disegni di ricerca}
La \textbf{ricerca fenomenologica}: descrive, interpreta e comprende i significati di esperienze vissute in prima persona. Si può condurre su piccoli gruppi o su individui, è essenziale che il ricercatore sia empatico, per poter trarre i dati.
La sua domanda è: "come si vive in una particolare situazione". Il campione si analizza con attenzione alla qualità, non alla quantità del gruppo, per evitare la ridondanza delle informazioni, ma cercando di avere una certa completezza. Potremmo utilizzare un singolo racconto, un evento irripetibile, o magari si può scegliere di analizzare un singolo caso di quelli che abbiamo analizzato, se tutti i casi danno informazioni molto simili. Una volta selezionato il campione, dobbiamo definire i parametri della ricerca, il tempo che vogliamo dedicarle, e quello che vogliamo ottenere. \'E fondamentale il consenso, e bisogna definire gli imprevisti che potremo incontrare, e come fargli fronte.
I dati si analizzano con empatia e intuizione. Se l'empatia manca, si può cercare di porre attenzione a determinati atteggiamenti, determinate espressioni.
\'E fondamentale non andare fuori tema, indirizzando le domande durante le interviste, e anche surante la stesura dell'elaborato dobbiamo mantenere presente il quesito, e rispettandolo. I dati non coerenti con il quesito iniziale, quindi, vanno eliminati. Alla fine, si deve ottenere una sintesi dei racconti e delle esperienze degli intervistati.

Nei risultati si presentano gli aspetti essenziali dell'esperienza, usando magari citazioni dei soggetti, descrivendo la modalità di presentazione e i temi essenziali dell'esperienza. Si crea una trama, inserendola in una struttura.

Un esempio potrebbe essere chiedersi quale sia il significato di vivere da soli per gli anziani con Aleheimer o demenze correlate, o per giovani con mielolesioni. 

La \textbf{grounded theory} vorrebbe sviluppare una teoria, descrivendo i problemi socio psicologici di base e i processi che avvengono in contesti sociali. Porta avanti allo stesso tempo la ricerca, la gestione, l'organizzazione e l'elaborazione dei dati raccolti. Tutti i passaggi della ricerca devono quinid avvenire contemporaneamente. Non c'è interruzione tra la raccolta e l'analisi dei dati, e questo consente una migliore dinamicità del processo: tutte le affermazioni possono essere mantenute o modificate alla luce dei dati raccolti, e il ricercatore partecipa al fenomeno, mentre lo studia. Le informazioni non vengono solo da dati formali, ma anche da interviste informali, che vengono ottenute in modo non convenzionale. Il campionamento deve essere molto più numeroso: entrando nel contesto captiamo tutto quello che possiamo captare, quindi più dati abbiamo, meglio è.
\'E possibile "perdersi" all'interno della mole di dati. Gli aspetti positivi sono la grande dinamicità e la possibilità di ottenere informazioni indirettamente, mentre quelli negativi sono appunto la possibilità di perdersi, e di perdere l'indirizzo iniziale della ricerca.

La \textbf{ricerca etnografica} è mirata ad indagare la cultura, i modi di agire, le dinamiche, le tradizioni, di un gruppo sociale, che abbia una organizzazione regolata, con ruoli e modalità coerenti e compatibili con le regole del sistema, che deve essere diverso dal nostro. Per farla, c'è bisogno di immergersi nell'ambiente che si vuole indagare. Si parte con dei contatti iniziali, con la revisione della letteratura, si raccolgono i dati relativi alla popolazione che vogliamo studiare, per poi confrontare nuovamente i dati raccolti con la letteratura.
\'E importante, in riabilitazione, per adeguare i trattamenti ai sempre più presenti immigrati.

Lo \textbf{studio di caso} è utilizzato quando vogliamo studiare una situazione complessa, unica e irripetibile, e quindi ci avvaliamo sia della quantitativa che della qualitativa, per questo disegno di ricerca. L'obiettivo è studiare l'insieme del paziente, sempre cercando di capire il "\textit{cosa}, \textit{come} e \textit{perché}", evitando di ridurre il fenomeno a poche variabili, approfondendo di più l'insieme. \'E utilizzato quando i confini tra il fenomeno e il contesto non siano così chiari. Le fonti sono più precise, grazie all'affiancamento della ricerca quantitativa, quindi l'organizzazione dei dati sarà specifica, e i dati verranno raccolti in maniera precisa. Bisogna identificare in maniera chiara il problema da comprendere, individuare domande e requisiti del nostro studio, individuare i componenti delle domande e i confini dello studio, se il campionamento non è significativo bisogna ripeterlo, bisogna predisporre un protocollo di analisi dei dati, e redarre un elaborato scientifico.

La scelta dei casi dipende dagli obiettivi che dobbiamo perseguire: essendo lo studio di caso un terreno comune alle due ricerche, la logica di campionamento è statistica, quindi ci servono abbastanza casi, scelti con criterio qualitativo, in modo che i soggetti rispondano in maniera soddisfacente ai quesiti, senza saturarci di informazioni o essere ridondanti. \'E anche possibile scegliere dei casi rappresentativi da un campione più ampio, che verrà sottoposto alla ricerca quantitativa. Il campionamento non può essere casuale, come nella quantitativa, ma deve essere intenzionale, effettuata secondo tre tipi di campionamento:
\begin{itemize}
\item Per massimizzare le differenze, prendiamo un caso critico, e indaghiamo le situazioni all'estremo.
\item Possiamo minimizzare le differenze
\item \textit{Snowball sampling} per restare il più possibile al centro della distribuzione dei casi
\end{itemize}

Anche la scelta delle tecniche è importante: questionari, interviste, test, focus group sono solo alcuni dei numerosi esempi.

La sfida di questo metodo è riuscire a fare andare bene sia la ricerca quantitativa che la qualitativa, quindi le informazioni devono andare bene per entrambe le ricerche.

Per padroneggiare i dati ci sono varie strategie: l'utilizzo di un database di primo livello, sintetizzando e codificando le informazioni, per dare una visione unitaria, a cui faccia seguito un database di secondo livello, per fare una sintesi di tutte le informazioni che abbiamo nel primo database, per dare anche un punto di vista scientifico alle nostre informazioni.

Tutte le interpretazioni dei dati che abbiamo raccolto vanno riportate, perché nella ricerca qualitativa l'interpretazione è molto importante. 

\section{Il campionamento}
Il campionamento è l'atto di selezione del campione, che è una piccola porzione della popolazione. Deve essere adeguato a studiare il fenomeno che ci interessa.
Bisogna selezionare i casi più ricchi di dati. Può essere eseguito prendendo i casi estremi, un campione omogeneo, un campione teorico, casuale, con una stratificazione intenzionale.
\subsection{Tipi di campionamento}
\begin{itemize}
\item Il campionamento teorico si usa nella textit{gruonded theory}, e in questa metodica le informazioni vengono codificate, si esegue un confronto e si creano delle memo. Se i dati sono incompleti o fanno emergere nuove ipotesi, si esegue un nuovo campionamento.
Il caso estremo: si trova ai lati della gaussiana.
\item La variazione massima può essere
\begin{itemize}
\item demografica: per sesso, razza, classe sociale, educazione, età e reddito.
\item fenomenica: variazione in base alla diagnosi e alla stadiazione
\item teoretica
\item contestuale: dove vivono i soggetti
\end{itemize}
\item Lo stratificato mirato: si usa quando si debba stratificare il campione esaminando vari aspetti. 
\end{itemize}
Gli obiettivi del processo di campionamento sono: identificare i criteri di inclusione e esclusione, identificare ulteriori specifici campioni, stimare la numerosità che vogliamo ottenere, e definire le strategie di reclutamento del campione.

La dimensione del campione va stimata in modo da avere informazioni utili, senza raggiungere la soglia di saturazione.
Per la Grounded Theory andiamo da 20 a 50, per la fenomenologica da 5 a 20, descrittivo interpretativo prevede circa 15 persone, e per l'etnografico non viene stimato il campione, ma bisogna avere una immersione intensa e partecipante.

Il campione può essere strutturato a palla di neve: si chiede al campoione chi sia molto informato su un certo aspetto, e reclutare così.
Può essere opportunistico, quindi individuare qualcuno che spieghi il concetto che ci interessa, o di convenienza, quindi cercando di reclutare chiunque possa avere informazioni utili.

\subsection{Come raccogliere le informazioni}
\subsubsection{Focus Group}
Il Focus group si basa sull'idea che l'interazione di un gruppo possa portare ad ottenere delle informazioni originali, che si basino sull'esperienza diretta dei partecipanti. Può essere utilizzato da solo, come tecnica qualitativa, o in associazione alla metodologia quantitativa.
\'E composto di tre fasi: in fase preliminare si identificano gli elementi costitutivi del problema, con un linguaggio adeguato per trovare i punti critici, la Ricerca verifica l'andamento della raccolta dati, per eventualemtne modificare il processo di raccolta, e in fase di Valutazion permette di verificare quali siano le conseguenze da trarre dalla ricerca, la reazione della popolazione a partire dai dati raccolti, e stabilire se l'interpretazione trova conferma presso la popolazione.
Con il focus group è importante definire che informazioni mi aspetto, come le utilizzerò, e per quqle scopo. Si conduce tramite una intervista, in modo da avere una linea guida per condurre l'incontro. Devo avere una scaletta di domande, in modo che io, moderatore, avrò sempre una idea di dove andremo a parare, per non perdere il filo, che è una cosa frequente, parlando in gruppo.
Questa guida non deve essere un questionario, non devo seguirlo esattamente domanda per domanda, ma in letteratura si dice che per avere una buona indagine, le domande debbano essere organizzate ad imbuto: partendo dal generale bisogna arrivare sempre più al particolare.
Il numero di partecipanti deve essere compreso tra circa 6 e circa 12, i criteri di inclusione vengono decisi in base al campionamento che abbiamo effettuato. Il ruolo del moderatore è fondamentale, per gestire la conversazione, con una particolare attenzione a cogliere tutte le idee che emergono, con un linguaggio e un comportamento adeguati, per stimolare e dirigere la conversazione, anche adattandosi al partecipante con cui interagiamo. La discussione deve restare sempre in tema, ma non deve essere troppo limitata dal moderatore. Dobbiamo restare neutrali, per non dirigere completamente il discorso, soprattutto non per le idee che ne devono emergere. Se il gruppo tende a far deviare il discorso, il moderatore deve riportare il discorso sui binari predefiniti. A mano a mano che si va avanti, le domande devono essere sempre più strutturate.

Luogo e durata: Va da 1 a 3 ore, con una location neutrale, ma ospitale, per mettere le persone a loro agio e farle parlare liberamente. Devono essere evitati focus group di persone che si conoscono, e che hanno gerarchie ben definite. \'E possibile creare dei focus group anonimi. sarebbe importante cercare di avere un luogo riservato e silenzioso. Si possono effettuare registrazioni audio e video, con il moderatore che intanto prende i suoi appunti, annotando i comportamenti non verbali dei partecipanti. La registrazione viene poi sbobinata, integralmente, e si selezionano le informazioni più importanti. Nelle conclusioni si possono inserire delle citazioni di quello che si è detto, purchè possano essere chiare e spiegare bene quello che si è detto.

Per capire se un focus group sia stato condotto bene o meno, è importante verificare come sono stati scelti i partecipanti, e quale sia la loro motivazione. Si controlla se l'ambiente era consono e organizzato in maniera adeguata, si controlla il disegno di ricerca, e il clima che si è instaurato nel focus group, quale sia stata la varietà delle idee emerse, e quali siano stati i tempi di conduzione e quelli riservati all'interazione dei partecipanti.

Si considera quello che viene fuori dalle nostre conclusioni, che devono essere chiare.

\subsubsection{Tecnica delphi}
\'E una tecnica strutturata, in cui gli esperti interagiscono non in modo diretto, ma in modo indiretto: gli esperti non devono essere presenti in un luogo definito, ma possono partecipare anche da casa. Si forniscono dei questionari a cui rispondere anonimamente, per rendere meno influenzate possibili le opinioni delle persone. Il ricercatore coordina e gestisce il gruppo, fornendo i questionari ai partecipanti, e costruendo poi dei questionari ulteriori sulla base delle risposte che vengono formulate. In questo caso, i partecipanti devonoe essere competenti ed esperti di un certo argomento. I partecipanti sono da 8 a 20, interagiscono a distanza, non sanno di conoscersi, quindi nessuno viene messo in soggezione. Le informazioni sono sempre raccolte tramite questionari o interviste scritte, che possono essere strutturate o non strutturate. Di solito la ricerca prende l'avvio con domande aperte, in cui ci si possa esprimere liberamente, e procede con domande chiuse, che sono più specfiche. Viene utilizzata anche la statistica, per defenire i risultati.
Per valutare l'accuratezza della tecnica delphi, si valuta
\begin{itemize}
\item La composizione degli esperti
\item La motivazione e puntualità nelle risposte (il materiale deve essere sempre conservato, per permettere una verifica della qualità della ricerca)
\item L'analisi dei dati
\item La pertinenza delle risposte
\item La tempestività delle risposte, o la difficoltà di ottenere le risposte nei tempi previsti
\item Il giudizio di chi ha partecipato alla ricerca.
\end{itemize}

\subsubsection{Intervista}
\'E una conversazione, sollecitata dall'intervistatore, che è anche il ricercatore, rivolta a soggetti selezionati con adeguato campionamento, in numero adeguato alle nostre finalità, guidata dall'intervistatore, sulla base di uno schema più o meno flessibile. Non è volta a ricavare dati, ma a comprendere la realtà che stimao studiando, in particolare come i soggetti che analizziamo possano vedere il mondo, e il fenomeno che ci interessa. L'obiettivo è fornire una cornice in cui gli intervistati possano definire il loro modo di sentire con le loro parole.
Il campione non è statisticamente significativo, quindi può essere ridotto, ma non deve fornire informazioni ridondanti, con un campionamento non probabilistico, ma con caratteristiche rilevanti per la nostra analisi. Il nostro scopo è ricostruire storie, non studiare variabili.

I vari tipi di intervista possono differenziarsi per il grado di standardizzazione, che sarebbe la libertà che si sceglie di concedere all'intervistato. L'intervista è utile come ausilio nella ricerca quantitativa nella fase pre, o come supporto alla quantitativa.

\begin{itemize}
\item L'intervista strutturata: a tutti gli intervistati sono poste le stesse domande, nella stessa formulazione e nella stessa sequenza, in maniera aperta. Cerca quinid di simulare l'approccio quantitativo, pur essendo qualitativo. \'E l'unica che può essere utilizzata negli studi sia qualitativi che quantitativi. Perde in flessibilità, perché le domande non possono essere guidate dalle risposte dell'interessato, ma possono essere studiate meglio. \'E comunque meno strutturata e standardizzata del questionario, perché le domande sono aperte e non chiuse. Non è il metodo migliore per conoscere il fenomeno in profondità, perché siamo legati a domande precise e non possiamo indagare oltre, ma ci permette di raccogliere dei dati oggettivi.

L'intervista strutturata viene preferita al questionario quando abbiamo situazioni molto variabili, perché permettiamo alla persona di dare risposte personali, o con un problema complesso, o grande, o quando la cultura dell'intervistato è poco adatta al linguaggio che potremmo utilizzare all'interno delle domande chiuse.

\item L'intervista semistrutturata: il ricercatore ha una traccia con gli argomenti da trattare nell'intervista, quindi non è completamente strutturata, nè completamente libera. C'è quindi una certa libertà, che permette di indagare al meglio gli argomenti, e di variare le domande, sia nella formulazione che nell'ordine.
Lo stile della conversazione, mentre nella strutturata resta sempre lo stesso per tutti, permette di essere regolato in base al contesto, anche spiegando meglio le domande. \'E possibile sviluppare temi che non erano in traccia, ma che ci interessano.

\item L'intervista non strutturata: l'intervistatore andrà a porre un tema generale, ma sarà l'intervistato a guidare la conversazione, scegliendo di cosa parlare e di cosa non parlare. Non può essere preparata in anticipo, ma ogni colloquio prenderà una piega diversa.
\end{itemize}

La scelta dell'intervista varia in base ai nostri obiettivi, se vogliamo, ad esempio, più descrivere o più comprendere il fenomeno. Se gli intervistatori sono molteplici, bisogna sceglierne una strutturata, per mantenere un minimo di obiettività, mentre nelle indagini con un solo intervistatore è possibile anche sceglierne di meno strutturate, mantenendola comunque.

L'intervista clinica è fortemente strutturata.

L'intervista va condotta spiegando al paziente cosa vorremmo ottenere dall'intervista, quindi quale sia lo scopo dell'intervista e della ricerca, perché abbiamo scelto lui come campione, e perché andremo a porre eventuali domande personali.
In seguito si fanno delle domande primarie, per introdurre il tema, che possono essere generiche, descrittive, o strutturali, o a contrasto. Esistono anche le domande sonda, che non sono vere e proprie domande, ma sono degli imput, che diamo al paziente per punzecchiarlo, senza imboccare delle risposte. Le domande sonda vengono strutturate ripetendo la domanda e parte della risposta, o mostrandosi molto interessati, o con un interesse particolare. Il linguaggio che dobbiamo utilizzare è il più possibile empatico, per cogliere le idee del paziente. Una coaa molto difficile è dirigere la conversazione senza dirigere troppo il pensiero di chi abbiamo davanti.

Dopo aver somministrato e registrato l'intervista, dobbiamo analizzare i dati raccolti, con attenzione anche al linguaggio non verbale.

I risultati vengono poi tradotti in una narrazione, raccogliendo gli episodi e le narrazioni di caso, inserendo eventualmente anche brani dalle interviste. La sintesi dei dati si effettua anche con delle analogie.

\section{Idee uscite in classe}
\begin{itemize}
\item Come adeguare la medicina occidentale alla visione tradizionale del gruppo cinese
\item Indagare sulla diversa percezione dei vaccini nei diversi gruppi etnici
\item Qual è il vissuto di malattia del caregiver nel trattamento del paziente con malattia cronico degenerativa
\item Differenti vissuti del paziente in base alla relazione impostata dal fisioterapista (secco, distaccato, empatico, neutro) e quale sia il migliore outcome riabilitativo
\item Rapporto tra regole sanitarie e infezioni post-operatorie
\item Il vissuto in riabilitazione del paziente sportivo in funzione al ritorno in campo.
\item Come il vissuto personale di malattia in una popolazione di anziani a ridotta mobilità influenza il recupero dalla patologia.
\end{itemize}